\documentclass[ a4paper,
%                draft,
                10pt, 
%                twocolumn,
		onecolumn,
%                nobalancelastpage,
%                aip,cha,
                aps,
%                reprint,
%                preprintnumbers,
                amsfonts,
                amssymb,
                amsmath,
                eqsecnum,
                nofootinbib,  % control position of footnote
                superscriptaddress]{revtex4}

\usepackage[colorlinks=true,linkcolor=blue]{hyperref}
\usepackage{docs}
\usepackage{bm}
\usepackage{graphicx}
\usepackage{bookmark}   % to display bookmarks when using xelatex
\usepackage{pifont}
% \usepackage{xeCJK}

\begin{document}
\title{Quick start with imcmc}

\author{Youhua Xu}
\email{yhxu@nao.cas.cn}
\affiliation{School of Physics, Nanjing University,
22 Hankou Road, Nanjing, Jiangsu 210093, China}
\affiliation{Key Laboratory of Space Astronomy and Technology,
National Astronomical Observatories,
CAS, Beijing 100012, China}

\date{\today}


\begin{abstract}
This document demonstrates how to use the MCMC library -- \textbf{imcmc}, with a few simple
examples.
\end{abstract}

\maketitle

\section{Why build a library?}
It's a nightmare when if you're managing a project with lots of source files.

\section{Basics: Bayesian Analysis $\&$ MCMC}
Before getting start to generate MC samples with imcmc, I will brieflly
review the basics of Bayesian analysis and MCMC (use Metropolis-Hastings
algorithm as the example)


\section{Algrithms}

\subsection{Affine-invariant ensemble}


\subsection{Metropolis-Hastings}


\subsection{Hamitonian-Monte Carlo}


\subsection{Nested sampling}

\section{Structure of the code}


\appendix
\section{Parser}
Usually one will use may parameters in her/his codes, which might be model parameters, names/paths of data and even precision
controlling parameters, thus a well-designed parser is very helpful.

\end{document}
